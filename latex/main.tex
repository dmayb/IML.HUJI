%%%%%%%%%%%%%%%%%%%%%%%%%%%%%%%%%%%%%%%%%
%  My documentation report
%  Objective: Explain what I did and how, in order to help someone continue with the investigation
%
% Important note:
% Chapter heading images should have a 2:1 width:height ratio,
% e.g. 920px width and 460px height.
%
% The images can be found anywhere, usually on sky surveys websites or the
% Astronomy Picture of the day archive http://apod.nasa.gov/apod/archivepix.html
%
% The original template (the Legrand Orange Book Template) can be found here --> http://www.latextemplates.com/template/the-legrand-orange-book
%
% Original author of the Legrand Orange Book Template:
% Mathias Legrand (legrand.mathias@gmail.com) with modifications by:
% Vel (vel@latextemplates.com)
%
% Original License:
% CC BY-NC-SA 3.0 (http://creativecommons.org/licenses/by-nc-sa/3.0/)
%%%%%%%%%%%%%%%%%%%%%%%%%%%%%%%%%%%%%%%%%
 
%----------------------------------------------------------------------------------------
%	PACKAGES AND OTHER DOCUMENT CONFIGURATIONS
%----------------------------------------------------------------------------------------

\documentclass[11pt,fleqn]{book} % Default font size and left-justified equations

\usepackage[top=3cm,bottom=3cm,left=3.2cm,right=3.2cm,headsep=10pt,letterpaper]{geometry} % Page margins

\usepackage{xcolor} % Required for specifying colors by name
\definecolor{ocre}{RGB}{52,177,201} % Define the orange color used for highlighting throughout the book

% Font Settings
\usepackage{avant} % Use the Avantgarde font for headings
%\usepackage{times} % Use the Times font for headings
\usepackage{mathptmx} % Use the Adobe Times Roman as the default text font together with math symbols from the Sym­bol, Chancery and Com­puter Modern fonts

\usepackage{microtype} % Slightly tweak font spacing for aesthetics
\usepackage[utf8]{inputenc} % Required for including letters with accents
\usepackage[T1]{fontenc} % Use 8-bit encoding that has 256 glyphs

% Bibliography
\usepackage[style=alphabetic,sorting=nyt,sortcites=true,autopunct=true,babel=hyphen,hyperref=true,abbreviate=false,backref=true,backend=biber]{biblatex}
\addbibresource{bibliography.bib} % BibTeX bibliography file
\defbibheading{bibempty}{}

\input{headers/structure} % Insert the commands.tex file which contains the majority of the structure behind the template

\newcommand{\R}{\mathbb{R}}
\newcommand{\E}{\mathbb{E}}
\newcommand{\P}{\mathbb{P}}

\newcommand{\X}{\mathcal{X}}
\newcommand{\Y}{\mathcal{Y}}
\newcommand{\H}{\mathcal{H}}
\newcommand{\N}{\mathcal{N}}

\newcommand{\e}{$\varepsilon$} % Insert mathematics and notations 

\begin{document}
\title{Introduction to Machine Learning (67577) - Course Book - Edition 1}

%----------------------------------------------------------------------------------------
%	TITLE PAGE
%----------------------------------------------------------------------------------------

\begingroup
\thispagestyle{empty}
\AddToShipoutPicture*{\put(0,0){\includegraphics[scale=1.25]{esahubble}}} % Image background
\centering
\vspace*{5cm}
\par\normalfont\fontsize{35}{35}\sffamily\selectfont
\textbf{Introduction to Machine Learning}\\
{\LARGE 67577\\Course Book}\par % Book title
\vspace*{1cm}
%{\Huge Andrea Hidalgo}\par % Author name
\endgroup

%----------------------------------------------------------------------------------------
%	COPYRIGHT PAGE
%----------------------------------------------------------------------------------------

\newpage
~\vfill
\thispagestyle{empty}
%\noindent Copyright \copyright\ 2014 Andrea Hidalgo\\ % Copyright notice
\noindent \textsc{Introduction to Machine Learning, The Hebrew University, Jerusalem}\\

\noindent \textsc{Address to code examples and labs repository}\\ % URL

\noindent Written by and by.\\ % License information

\noindent \textit{First release, October 2020} % Printing/edition date

%----------------------------------------------------------------------------------------
%	TABLE OF CONTENTS
%----------------------------------------------------------------------------------------

\chapterimage{head1.png} % Table of contents heading image
\pagestyle{empty} % No headers
\tableofcontents % Print the table of contents itself
\cleardoublepage % Forces the first chapter to start on an odd page so it's on the right
\pagestyle{fancy} % Print headers again

\chapter{Mathematical Basis}
    \section{Linear Algebra}
        \subsection{Hyperplanes}
        \subsection{Projecting Matrices}
        \subsection{Matrix Decomposition}
            \subsubsection{Eigenvalues Decomposition}
            \subsubsection{Singular Values Decomposition}
    \section{Calculus}
        \subsection{High Order Derivatives}
        \subsection{Convexity}
    \section{Probabilities Theory}
        \subsection{Distributions and Random Variables}
        \subsection{Multi-variate Distributions}
        \subsection{Joint- and Marginal Distributions}
        \subsection{PDF and CDF of Distributions}
        \subsection{Measurements Of Concentration}

\chapter{Introduction \& Linear Regression}
    \section{Introduction to Statistical Learning}
        \subsection{Estimation Theory}
            \subsubsection{Estimators}
            \subsection{Risk \& Loss Functions}
        \subsection{Learning Principals}
            \subsubsection{Empirical Risk Minimization}
            \subsubsection{Maximum Likelihood}
        \subsection{Lab: Python Data Analysis - First Steps}
        \subsection{Lab: Data Simulation and Sampling}
        
    \section{Linear Regression}
        \subsection{Ordinary Least Squares}
        \subsection{Weighted Least Squares}
        \subsection{Geometric Interpretation}
        \subsection{Categorical Variables}
        \subsection{Lab: Linear Regression}
        
    \section{Beyond Linearity}
        \subsection{Polynomial Fitting}
        \subsection{Poisson Regression}
        \subsection{Lab: Polynomial Fitting}
    
\chapter{Classification}
    \section{Classification Overview}
        \subsection{Loss Function}
        \subsection{Type-I and Type-II Errors}
        \subsection{Statistical Measures of Performance}
        
    \section{Logistic Regression}
        \subsection{A Probabilistic Model For Noisy Labels}
            \subsubsection{The Hypothesis Class}
            \subsubsection{Learning Via Maximum Likelihood}
        \subsection{Computational Implementation}
        \subsection{Interpretability}
        \subsection{ROC Curve}
        \subsection{Lab: Logistic Regression}
    
    \section{Half-Space Classifier}
        \subsection{Learning Linearly Separable Data Via ERM}
        \subsection{Computational Implementation}
        \subsection{The Perceptron Algorithm}
        
    \section{Support Vector Machines}
        \subsection{Maximum Margin Learning Principal}
        \subsection{Hard-SVM}
        \subsection{Soft-SVM}
            \subsubsection{The Kernel Trick}
        
    \section{Nearest Neighbors}
        \subsection{Graph-Based Approach For Learning}
        \subsection{Classification \& Regression Using $k$-NN}
        \subsection{Computational Implementation}
        \subsection{Selecting Value of $k$ Hyper-Parameter}
        \subsection{Variants of Nearest Neighbors}
        
    \section{Decision Trees}
        \subsection{Axis-Parallel Partitioning of $\R^d$}
        \subsection{Classification \& Regression Trees}
        \subsection{Growing a Classification Tree}
        \subsection{NP-Hardness and CART Heuristic}
        \subsection{Pruning a Decision Tree}
    
    \section{Bayes Classifiers}
        \subsection{Bayes Optimal Classifier}
        \subsection{Naive Bayes}
        \subsection{Linear Discriminant Analysis}
        \subsection{Quadratic Discriminant Analysis}
        \subsection{Lab: Maximum Likelihood Estimation}
        
\chapter{PAC Theory of Statistical Learning}
    \section{Theoretical Framework For Learning}
        \subsection{Data-Generation Model}
        \subsection{The Realizability Assumption}
        \subsection{Learning As A Game - First Attempt}
        \subsection{Probably- and Approximately Correct Learners}
    \section{No Free Lunch and Hypothesis Classes}
        \subsection{No Free Lunch!}
        \subsection{Restricting for Hypothesis Classes}
        \subsection{Learning As A Game - Final Attempt}
        \subsection{Example: Threshold Functions}
    
    \section{PAC Learnability of Finite Hypothesis Classes}
    
    \section{VC-Dimension}
        \subsection{Formal Definition}
        \subsection{VC-Dimension of Finite Hypothesis Classes}
        \subsection{Example: Axis Aligned Rectangles}
        \subsection{Example: Half-Spaces}
        
    \section{Agnostic PAC: Extending Framework}
        \subsection{Data-Generation Model Over \X\times\Y}
        \subsection{Relaxing Realizability Assumption}
        \subsection{Introducing General Loss Functions}
        \subsection{Agnostic PAC Learnability}
    
    \section{Uniform Convergence Property}
        \subsection{\e-Representative Datasets}
        \subsection{Achieving Uniformity In \H and \D}
            \subsubsection{The Case Of Finite \H}
            \subsubsection{The General Case - Infinite \H}
    
    \section{The Fundamental Theorem of Statistical Learning}
    
\chapter{Ensemble Methods}
    \section{Bias-Variance Trade-off}
        \subsection{Generalization Error Decomposition}
        \subsection{Lab: Bias-Variance Via Decision Trees}
        \subsection{Lab: Bias-Variance Via Polynomial Fitting}
        
    \section{Ensemble/Committee Methods}
        \subsection{Weak-Learnability}
        \subsection{Uncorrelated Predictors}
        \subsection{Correlated Predictors}
        \subsection{Committee Methods In Machine Learning}
    
    \section{Boosting Weak-Learners}
        \subsection{AdaBoost Algorithm}
        \subsection{Gradient Boosting Algorithm}
        \subsection{Lab: Boosting - Image Classification}
        
    \section{Bagging}
        \subsection{Bootstrapping}
            \subsubsection{} % wider use
        \subsection{Bagging Reduces Variance}
        \subsection{Random Forests Bagging and De-correlating Decision Trees}
        
\chapter{Regularization, Model Selection and Model Evaluation}
    \section{Regularization}
        \subsection{Best Subset Selection}
        \subsection{$L_q$ Nrom Regularizes}
            \subsection{Ridge Regularization}
            \subsection{Convexity vs. Sparsity}
            \subsection{Lasso Regularization}
        \subsection{Lab: Regularized Logistic Regression}
    
    \section{Model Selection and -Evaluation}
        \subsection{Cross Validation}
        \subsection{Bootstrap}
        \subsection{Common Model Selection Mistakes}
            \subsubsection{Over-estimating Generalization Error}
            \subsubsection{Under-estimating Generalization Error}
    \subsection{Lab: Selecting Regularized Model}
        
\chapter{Unsupervised Learning}
    \section{Dimensionality Reduction}
        \subsection{Preserved Data Properties}
        \subsection{Principal Component Analysis}
            \subsubsection{Closest Subspace Interpretation}
            \subsubsection{Generalized Linear Regression Interpretation}
            \subsubsection{Maximum Retained Variance Interpretation}
            \subsubsection{Projection- vs. Coordinates of Data-Points}
            \subsubsection{Variants of PCA}
        \subsubsection{Euclidean Embedding}
        \subsection{Lab: PCA}
        
    \section{Clustering}
        \subsection{K-Means}
            \subsubsection{}
            \subsubsection{K-Means++}
        \subsection{Mixture of Gaussians}
            \subsubsection{Expectation Minimization Learning Principal}
            \subsubsection{Estimating Model Parameters}
        \subsection{Spectral Clustering}
        \subsection{Lab: K-Means++}
        \subsection{Lab: Parameters Estimation In MoG}
    
\chapter{Convex Optimization and Gradient Descent}
    \subsection{Gradient Descent Learning Principal}
    \subsection{Utilizing Sub-gradients For GD}
    \subsection{Stochastic Gradient Descent}
    \subsection{Variants Of Gradient Descent}
        \subsection{Initialization Conditions}
        \subsection{Tuning Learning Rates}

\chapter{Online- and Reinforcement Learning}
\chapter{Deep Learning}

\newpage
\end{document}